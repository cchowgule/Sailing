\documentclass[12pt]{scrartcl}

\usepackage{hyperref}
\usepackage{tabularx}
\usepackage{pdflscape}
\usepackage{caption}
\usepackage{verbatim}
\usepackage{multirow}

\hypersetup{
	pdftitle={Basic Sailng 1 - Course Material},
	pdfauthor={Chaitanya Chowgule},
	colorlinks=true,
	linkcolor=blue
}

\setcounter{tocdepth}{2}

\title{Basic Sailing 1}
\subtitle{Course Material}
\date{}

\begin{document}

\maketitle

\thispagestyle{empty}

\tableofcontents

\newpage

\section{Outline} \label{sec:outline}

This course aims to introduce complete novices to the most fundimental sailing techniques such as steering, sail trim, tacks, and gybes. It is perhaps the most important as without these fundimentals future courses will be much harder to teach. The aim is to build a strong base and ensure that new sailors do not develop bad habbits that may make it harder for them to sail in more challenging conditions. The course will be conducted on a reach-to-reach course about 8 boatlengths long.

\subsection{Course Goals} \label{subsec:goals}

Participants should leave the course having mastered:

\label{list:goals}
\begin{enumerate}
	\item Identifying points of sail
	\item Identifying major roles on a Seabird
	\item Steering to a course or to a sail setting
	\item Trimming to a course or a point of sail
	\item Controlled stops and starts
	\item Basic tack without getting stuck in irons
	\item Basic gybe while maintaining control of direction and sail
\end{enumerate}

\subsection{Course Requirements} \label{subsec:requirements}

\label{tab:requirements}
\begin{tabular}{|l|l|}
	\hline
	Condtions & \hyperlink{condition:light}{Light Day} \\
	\hline
	Grade & \hyperlink{grade:trainee}{Trainee} \\
	\hline
	Maximum Participants & 4 \\
	\hline
\end{tabular}

\section{Schedule} \label{sec:schedule}

\label{tab:schedule}
\begin{tabularx}{\textwidth}{|X|X|}
	\hline
	\textbf{Time} & \textbf{Activity} \\
	\hline
	10 mins. & Setup for course before participants arrive. Sign out all materials needed for the course and make sure they are in working order. Load materials needed on the water on the powerboat and confirm it is fueled \\
	\hline
	30 mins. & Briefing: \\
	& Get to know the paticipants. Names, when the did the previous courses, etc. \\
	& Refresh skills and info from previous courses relevant to this course. \\
	& List all skills to be covered today and go over the sailing area and the course. \\
	& Go through sklls individually paying careful attention to common mistakes and how to correct them. \\
	& Explain the test and its criteria for passing. \\
	\hline
	15 mins. & Drop participants to the boat. Setup the training course. \\
	\hline
	1 hrs. 30 mins. & Go through the skills. Have them rotate roles every 4 completed circuits. \\
	\hline
	30 mins. & Conduct the test. \\
	\hline
	15 mins. & Have the participants return to the mooring. Pick up the course and return to land. \\
	\hline
	30 mins. & Debriefing: \\
	& Recap all skills covered on the water \\
	& Point out common mistakes and single out those who corrected them successfully. Explain how they did so. \\
	\hline
	10 mins. & Clear the powerboat and briefing area. Return all items and sign them back in. \\
	\hline
\end{tabularx}
\\
\\
Total time: 4 hrs. 20 mins.

\newpage

\section{Syllabus} \label{sec:syllabus}

\subsection{Setup} \label{subsec:setup}

Before the participants arrive check that the weather conditions acceptable for the course and decide on the sailing area for the day.
\\
\\
Sign out the following:

\label{list:materials}
\begin{itemize}
	\item Whiteboard, markers and figures
	\item 2 marks with anchors and anchor lines
	\item Wind indicator
\end{itemize}
\\
Load the marks on to the powerboat and ensure there is enough fuel for the day.
\\
\\
Set up the whiteboard for the briefing.

\subsection{Briefing} \label{subsec:briefing}

Gather the participants in front of the whiteboard, take down their names, and confirm that they have completed the relevant prerequisite courses. If someone has mistakenly come to a course they are not yet qualified to attend inform them which course they should be attending and if possible help them schedule a time to complete it.
\\
\\
On the whiteboard illustrate where on the river the course will be taking place, show the orientation of the marks in relation to some landmark, and if possible point it out from the briefing location.
\\
\\
Start by refreshing material from previous courses. Go over finding the wind direction, estimating wind speed, parts of the boat and their names, leaving and coming to the mooring, man-overboard drills, etc., briefly****.
\\
\\
List out all the skills that will be covered during the course, very briefly exapanding on each one.
\\
\\
Go down the list of skills explaining the salient points of each one:

\subsubsection{Identifying points of sail} \label{subsubsec:points-of-sail}

Using the points of sail diagram, illustrate the various points of sail and their names. Describe in general how the sails are loosened as the boat is steered away from the wind. Go over the no go zone and how to sail upwind by going from tack to tack. Using the diagram explain how tacks take the bow throuh the wind causing the sail to change sides and gybes take the stern through the wind with the same result. Then add in an illustration of the days course showing how the marks will create a reach-to-reach course. Point out where the boat will have to tack or gybe to round the course.

\subsubsection{Identifying major roles on a Seabird} \label{subsubsec:major-roles}

Transition to the roles on the Seabird and how the changes in course will be handled by each. Start with the helm, the main sheet trimmer, the person responsible for the runners and centreboard and finally the jib trimmer.
\\
\\
Draw and illustration of the Seabird from a top view and mark where each person should be sitting. Detail how the should cross the boat during a tack or a gybe.

\subsubsection{Steering to a course or sail setting \& trimming to a course or a point of sail} \label{subsubsec:steering-trimming}

Introduce the 2 major modes of steering. Explain that when steering to a course the helm maintains a steady heading while the trimmers adjust the sails to the course while when steering to a point of sail the trimmers set the sails and the helm must maintain boat speed by adjusting the heading.

\subsubsection{Controlled starts and stops} \label{subsubsec:starts-stops}

Paying close attention to the difference between being in irons and being in a controlled stop, go over the procedure for coming to a stop, maintaining a stationary position, and building boat speed to come out of a stop.

\subsubsection{Basic tacks without getting stuck in iron} \label{subsubsec:tacks}

\subsubsection{Basic gybes while maintaining control of direction and sail} \label{subsubsec:gybe}

\subsubsection{Explain the test and its criteria for passing} \label{subsubsec:test}

\begin{comment}
	%TODO
	Syllabus --- special note setup for teaching tips
		setup
		on water
		packup
		debrief
		packup
	Test --- with checkboxes and final tabulation
	Diagrams & Teaching Aids --- on individual pages for ease of printing
		points of sail
		course
		handouts

\end{comment}

\section{Course Goals} \label{sec:goals}

Example

\begin{enumerate}
	\item Example
\end{enumerate}

Example \hyperlink{example:example}{Example}

\newpage

\begin{landscape}

	Example

	\begin{center}
		\scriptsize{
			\begin{tabularx}{520pt}{|X|}
				\hline
				Example \\
				\hline
			\end{tabularx}
		}
	\end{center}

	Example

\end{landscape}

\newpage

\begin{tabularx}{\textwidth}{|X|}

	\hline
	\hypertarget{example:example}{Example} \\
	\hline

\end{tabularx}

\subsection{Example} \label{subsec:example}

\begin{tabular}{|l|}
	\hline
	Example \\
	\hline
\end{tabular}

\nameref{subsec:example}

\end{document}
