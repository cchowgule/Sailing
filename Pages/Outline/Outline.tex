\documentclass[12pt]{scrartcl}
\usepackage{hyperref}
\usepackage{tabularx}
\usepackage{pdflscape}
\usepackage{caption}
\usepackage{verbatim}
\usepackage{multirow}
\usepackage{ltablex}
\usepackage{enumitem}
\usepackage{amssymb}
\usepackage{graphicx}
\usepackage{float}
\usepackage[parfill]{parskip}

\hypersetup{
	pdftitle={Goa Yachting Association Training Outline},
	pdfauthor={Chaitanya Chowgule},
	colorlinks=true,
	linkcolor=blue
}

\setcounter{tocdepth}{2}

\title{GYA Training Outline}
\date{}

\newlist{test}{itemize}{1}
\setlist[test]{label=$\square$}

\graphicspath{{/media/interchange/WD/Sailing/PNGs/}}

\makeatletter
\renewcommand{\@seccntformat}[1]{}
\makeatother

\captionsetup[figure]{labelformat=empty, textfont=it}

\begin{document}

\maketitle

\thispagestyle{empty}

\tableofcontents

\newpage

\section{Preface} \label{sec:preface}

GYA will need to do some groundwork to set this program up. The club will require the following:

\label{list:club requirements}
\begin{itemize}
	\item 3 - 5 Wind speed and direction indicators
	\item 6 - 8 marks with sufficiently long anchorlines and anchors
	\item A Ledger of all members kept at the jetty
	\item A whiteboard, markers, and a set of boat, mark, and wind models
	\item 15 - 20 m of rope for teaching purposes
	\item A location where land based courses, briefings, and debriefings can be conducted
\end{itemize}

The club will also have to train all the tindles till at least the level of \hyperlink{grade:intermediate}{intermediate} skipper.

The club will also need to compile a list of emergency phone numbers such as the nearest ambulance service, nearest emergency medical treatment unit, Coastal Police, Coast Guard, etc. Every day the club should send out a WhatsApp message informing members of the day’s weather conditions. When members call to book a boat they should also have to mention what their grade is and the person taking the booking should confirm on the day if they have the requisite grade to skipper the boat that day and what restrictions they will be under.

Training should be free (except for boat rental charges) and conducted by volunteer members. Hiring a coach or paying for training will lessen the incentive for new members and training by experienced members will help build relationships between members and increase their investment in the club. Training consists of some number of classes taken by a trained member, the participant can then practice for as long as they like, then they take a test with a trained member.

New members with sailing experience do not need to start at the beginning. After discussing their experience with an experienced member they can skip as many grades as their experience allows. They must do at least the tests for the courses to achieve an \hyperlink{grade:intermediate}{Intermediate Grade}.

The club should have every member sign an indemnifying document clearing the club of any liability in the event of material damage, injury, or death.

All boats should be insured against damage and third party liability. The cost of this insurance should be recouped with a damage waiver charge on members. Members should sign a document indicating they have paid the damage waiver and therefore are not liable for damage to the boat or any third party damage up to some predetermined limit.

This system is based on my experiences learning to sail from a wide variety of coaches and teaching others to sail. It has also been adapted from Community Boating Inc.’s system in Boston. The advantages of this system is that it is cost effective, it is a simple way to make sure members are trained in basic safety procedures and are only taking boats out in conditions which are safe for them and the boats. The added advantage is that it gives members a clear trajectory of courses so they can track their progress and feel like they are achieving something as they learn to sail.

When the club purchases or is given more boats new skipper grades can be introduced so that the club can track which members are able to safely sail different boats. For example a member may be an \hyperlink{grade:expert}{expert} skipper on a Seabird but only a \hyperlink{grade:beginner}{beginner} skipper on a Laser. In Boston, CBI has a swipe card system where each card holds each member’s various grades and levels. This can be something the club aims at but with only 1 type of boat available right now it isn’t necessary. They have also gotten their members very invested in the club, all boat maintenance, rescue, administrative work, and race committees are done by club members who volunteer their time.

Having tindles accompany all members when they take out a boat or insist that members wear life jackets while on a Seabird is really a waste of the tindles and doesn’t actually make the chance of something going wrong any less. What the club needs is a way of making sure that members taking boats out are, at the very least, trained in safety procedures and are only taking the boat out in conditions that they are capable of sailing in.

\begin{landscape}

Keeping track of members and their grades can be done through a ledger with the following columns:

\label{tab:ledger}
\begin{tabularx}{670pt}{|X|X|X|X|X|X|X|X|}
	\hline
	Name & & & Trainee & & Beginner & & Intermediate \\
	\hline
	& Rigging and \newline Land Skills & Basic \newline Sailing 1 & Safety \newline Procedures & Basic \newline Sailing 2 & Intermediate Sailing & Spinnaker Drill & Advanced Sailing \\
	& & & & & & & \\
	& Date \& \newline Signature &  Date \& \newline Signature & Date \& \newline Signature & Date \& \newline Signature & Date \& \newline Signature & Date \& \newline Signature & Date \& \newline Signature \\
	\hline
	& & & & & & & \\
\end{tabularx}

The member administering the test signs and dates when the member taking the test has passed.

\end{landscape}

\section{Weather Conditions} \label{sec:weather conditions}

\label{tab:weather conditions}
\begin{tabularx}{\textwidth}{|X|X|X|}
	\hline
	& Max. Wind & Avg. Wind \\
	\hline
	\hypertarget{condition:light day}{Light Day} & \textless 16 kts & \textless 12 kts \\
	\hline
	\hypertarget{condition:fair day}{Fair Day} & \textless 20 kts & \textless 16 kts \\
	\hline
	\hypertarget{condition:heavy day}{Heavy Day} & \textless 24 kts & \textless 20 kts \\
	\hline
\end{tabularx}

Restrictions and requirements are based on the day’s weather conditions. Each condition can be given a name or a colour to make it easier to identify.

\newpage

\begin{landscape}

	\section{Skipper Grades} \label{sec:skipper grades}

	\label{tab:skipper grades}
	\begin{tabularx}{670pt}{|X|X|X|X|X|}
		\hline
		& \hypertarget{grade:trainee}{\bfseries{Trainee}} & \hypertarget{grade:beginner}{\bfseries{Beginner}} & \hypertarget{grade:intermediate}{\bfseries{Intermediate}} & \hypertarget{grade:expert}{\bfseries{Expert}} \\
		\hline
		\bfseries{Restrictions:} & & & & \\
		\hline
		\hyperlink{condition:light day}{Light Day} & No Spinnaker \newline Tindle Required & No Spinnaker & None & None \\
		\hline
		\hyperlink{condition:fair day}{Fair Day} & Not Allowed & No Spinnaker \newline Tindle Required & No Spinnaker & None \\
		\hline
		\hyperlink{condition:heavy day}{Heavy Day} & Not Allowed & Not Allowed & No Spinnaker \newline Tindle Required & No Spinnaker \\
		\hline
		\bfseries{Courses Required} & \nameref{subsec:rigging and land skills} \newline \nameref{subsec:basic sailing 1} & \nameref{subsec:safety procedures} \newline \nameref{subsec:basic sailing 2} & \nameref{subsec:intermediate sailing} \newline \nameref{subsec:spinnaker drill} & \nameref{subsec:advanced sailing} \\
		\hline
		\bfseries{Instructor Courses} & None & \multicolumn{3}{|c|}{\nameref{subsec:powerboat operation}} \\
		& & \multicolumn{3}{|c|}{\nameref{subsec:teaching tips and techniques}} \\
		\hline
	\end{tabularx}

\end{landscape}

\section{General Courses} \label{sec:general courses}

Each course can have a maximum of 4 participants per session.

\subsection{Rigging and Land Skills} \label{subsec:rigging and land skills}

\label{tab:rigging and land skills:requirements}
\begin{tabular}{ll}
	Requirements: & \hyperlink{condition:light day}{Light Day} \\
\end{tabular}

\subsubsection{Outline} \label{subsubsec:rigging and land skills:outline}

\begin{itemize}
	\item 1 hr. land session consisting of finding wind direction, estimating wind strength, determining tide direction, basic knots and a briefing of all skills to be covered on the water.
	\item 2 hr. session on the boat at mooring consisting of rigging and derigging both seabirds and safe casting off and coming to the mooring.
	\item 30 min. debrief on land going over all skills learnt that day.
\end{itemize}

\subsubsection{Test} \label{subsubsec:rigging and land skills:test}

\label{tab:rigging and land skills:test:requirements}
\begin{tabular}{ll}
	Requirements: & \hyperlink{condition:light day}{Light Day} \\
\end{tabular}

\begin{itemize}
	\item 30 - 45 mins. covering all skills. Each participant must perform casting off and coming to the mooring at the helm.
\end{itemize}

\subsection{Basic Sailing 1} \label{subsec:basic sailing 1}

\label{tab:basic sailing 1:requirements}
\begin{tabular}{ll}
	Requirements: & \hyperlink{condition:light day}{Light Day} \\
\end{tabular}

\subsubsection{Outline} \label{subsubsec:basic sailing 1:outline}

\begin{itemize}
	\item 30 min. briefing of all skills to be covered on the water.
	\item 2 hr. session on the boat consisting of reach to reach practice of steering, tacks and gybes between 2 marks.
	\item 30 min. debrief on land going over all skills learnt that day.
\end{itemize}

\subsubsection{Test} \label{subsubsec:basic sailing 1:test}

\label{tab:basic sailing:test:requirements}
\begin{tabular}{ll}
	Requirements: & \hyperlink{condition:light day}{Light Day} \\
\end{tabular}

\begin{itemize}
	\item 30 - 45 mins. covering all skills. Each participant must perform 4 tacks and 4 gybes at the helm.
\end{itemize}

\subsection{Safety Procedures} \label{subsec:safety procedures}

\label{tab:safety procedures:requirements}
\begin{tabular}{ll}
	Requirements: & \hyperlink{condition:light day}{Light Day} \\
	& \hyperlink{grade:trainee}{Trainee Grade} \\
\end{tabular}

\subsubsection{Outline} \label{subsubsec:safety procedures:outline}

\begin{itemize}
	\item 30 min. briefing of all skills to covered on the water, emergency phone numbers, and dangerous areas in the Zuari from the bridge to the mouth.
	\item 2 hr. session on the boat consisting of emergency stops, dropping anchor, man-overboard drills, reefing the sail, and catastrophic equipment failure.
	\item 30 min. debrief on land going over all skills learnt that day.
\end{itemize}

\subsubsection{Test} \label{subsubsec:safety procedures:test}

\label{tab:safety procedures:test:requirements}
\begin{tabular}{ll}
	Requirements: & \hyperlink{condition:light day}{Light Day} \\
\end{tabular}

\begin{itemize}
	\item 30 - 45 min. covering all skills. Each participant must complete a man-overboard drill at the helm.
\end{itemize}

\subsection{Basic Sailing 2} \label{subsec:basic sailing 2}

\label{tab:basic sailing 2:requirements}
\begin{tabular}{ll}
	Requirements: & \hyperlink{condition:light day}{Light Day} \\
	& \hyperlink{grade:trainee}{Trainee Grade} \\
	& \nameref{subsec:safety procedures} \\
\end{tabular}

\subsubsection{Outline} \label{subsubsec:basic sailing 2:outline}

\begin{itemize}
	\item 30 min. briefing of all skills to be covered on the water.
	\item 3 hr. session on the boat consisting of beat to run practice of steering, tacks and gybes between 3 marks.
	\item 30 min. debrief on land going over all skills learnt that day.
\end{itemize}

\subsubsection{Test} \label{subsubsec:basic sailing 2:test}

\label{tab:basic sailing 2:test:requirements}
\begin{tabular}{ll}
	Requirements: & \hyperlink{condition:light day}{Light Day} \\
\end{tabular}

\begin{itemize}
	\item 30 - 45 mins. covering all skills. Each participant must complete the windward leeward course 3 times at the helm.
\end{itemize}

\subsection{Intermediate Sailing} \label{subsec:intermediate sailing}

\label{tab:intermediate sailing:requirements}
\begin{tabular}{ll}
	Requirements: & \hyperlink{condition:fair day}{Fair Day} or lighter \\
	& \hyperlink{grade:beginner}{Beginner Grade} \\
\end{tabular}

\subsubsection{Outline} \label{subsubsec:intermediate sailing:outline}

\begin{itemize}
	\item 30 min. planning a 3 hr. sail based on wind and tide conditions and a briefing of all skills to be covered on the water.
	\item 3 hr. session on the boat consisting of trimming and boat handling and keeping a course with, against, and across the tide.
	\item 30 min. debrief on land going over all skills learnt that day.
\end{itemize}

\subsubsection{Test} \label{subsubsec:intermediate sailing:test}

\label{tab:intermediate sailing:test:requirements}
\begin{tabular}{ll}
	Requirements: & \hyperlink{condition:fair day}{Fair Day} \\
\end{tabular}

\begin{itemize}
	\item 1 hr. covering all skills. Each participant must sail a beat holding a steady course at the helm and trim both sails on a downwind course.
\end{itemize}

\subsection{Spinnaker Drill} \label{subsec:spinnaker drill}

\label{tab:spinnaker drill:requirements}
\begin{tabular}{ll}
	Requirements: & \hyperlink{condition:fair day}{Fair Day} or lighter \\
	& \hyperlink{grade:beginner}{Beginner Grade} \\
	& \nameref{subsec:intermediate sailing} \\
\end{tabular}

\subsubsection{Outline} \label{subsubsec:spinnaker drill:outline}

\begin{itemize}
	\item 30 min. briefing of all skills to be covered on the water.
	\item 3 hr. session on the water consisting of setting, trimming, gybing and breaking the spinnaker.
	\item 30 min. debrief on land going over all skills learnt that day.
\end{itemize}

\subsubsection{Test} \label{subsubsec:spinnaker drill:test}

\label{tab:spinnaker drill:test:requirements}
\begin{tabular}{ll}
	Requirements: & \hyperlink{condition:fair day}{Fair Day} \\
\end{tabular}

\begin{itemize}
	\item 30 - 45 mins. covering all skills. Each participant must perform each position in the spinnaker drill.
\end{itemize}

\subsection{Advanced Sailing} \label{subsec:advanced sailing}

\label{tab:advanced sailing:requirements}
\begin{tabular}{ll}
	Requirements: & \hyperlink{condition:heavy day}{Heavy Day} or lighter \\
	& \hyperlink{grade:intermediate}{Intermediate Grade} \\
\end{tabular}

\subsubsection{Outline} \label{subsubsec:advanced sailing:outline}

\begin{itemize}
	\item 30 min. briefing of all skills to be covered on the water.
	\item 3 hr. session on the boat consisting of heavy wind sailing techniques.
	\item 30 min. debrief on land going over all skills learnt that day.
\end{itemize}

\subsubsection{Test} \label{subsubsec:advanced sailing:test}

\label{tab:advanced sailing:test:requirements}
\begin{tabular}{ll}
	Requirements: & \hyperlink{condition:heavy day}{Heavy Day} \\
\end{tabular}

\begin{itemize}
	\item 2 - 3 hrs. covering all skills.
\end{itemize}

\newpage

\section{Instructor Courses} \label{sec:instructor courses}

Each course can have a maximum of 6 participants per session.

\subsection{Powerboat Operation} \label{subsec:powerboat operation}

\label{tab:powerboat operation:requirements}
\begin{tabular}{ll}
	Requirements: & \hyperlink{grade:beginner}{Beginner Grade} \\
\end{tabular}

\subsubsection{Outline} \label{subsubsec:powerboat operation:outline}

\begin{itemize}
	\item 30 min. briefing of all skills to be covered on the water.
	\item 1 hr. 30 min. session on a powerboat covering safe operation and safe rescue procedures.
	\item 30 min. debrief on land going over all skills.
\end{itemize}

\subsection{Teaching Tips and Techniques} \label{subsec:teaching tips and techniques}

\label{tab:teaching tips and techniques:requirements}
\begin{tabular}{ll}
	Requirements: & \hyperlink{grade:beginner}{Beginner Grade} \\
	& \nameref{subsec:powerboat operation} \\
\end{tabular}

\subsubsection{Outline} \label{subsubsec:teaching tips and techniques:outline}

\begin{itemize}
	\item 30 min. session covering basic teaching techniques, common errors by participants and their corrective measures, and test taking.
\end{itemize}

\end{document}
